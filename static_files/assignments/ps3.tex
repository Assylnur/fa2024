\documentclass[11pt, oneside]{article}   	% use "amsart" instead of "article" for AMSLaTeX format
\usepackage{geometry}                		% See geometry.pdf to learn the layout options. There are lots.
\geometry{letterpaper}                   		% ... or a4paper or a5paper or ... 
%\geometry{landscape}                		% Activate for rotated page geometry
\usepackage[parfill]{parskip}    	% Activate to begin paragraphs with an empty line rather than an indent
\usepackage{graphicx}				% Use pdf, png, jpg, or eps§ with pdflatex; use eps in DVI mode
								% TeX will automatically convert eps --> pdf in pdflatex		
\usepackage{amssymb}
\usepackage{amsmath}

%SetFonts

%SetFonts


\title{Problem Set 3 -- Shallow Networks}
\author{DS 542 -- DL4DS}
\date{Fall 2024}

\begin{document}
%\vspace*{-20pt}
\maketitle

Figures and equations referenced are in the book Understanding Deep Learning.

\textbf{Problem 3.1} 
What kind of mapping from input to output would be created if the activation
function in equation 3.1 was linear so that \(a[z] = \psi_0 + \psi_1 z\)?

\vspace{1cm}

\textbf{Problem 3.2} For each of the four linear regions in figure 3.3j, 
indicate which hidden units are inactive and which are active (i.e., which do
and do not clip their inputs).

\vspace{1cm}

\textbf{Problem 3.3} Prove that the following property holds for
\(\alpha \in \Re^+\):

\[\mathrm{ReLU}[\alpha \cdot z] = \alpha \cdot \mathrm{ReLU[z]}.\]

This is known as the non-negative homogeneity property of the ReLU function.

\vspace{1cm}

\textbf{Problem 3.4} (a) Following on from problem 3.3, what happens to the
shallow network defined in equations 3.3 and 3.4 when we multiply the
parameters \(\theta_{10}\) and \(\theta_{11}\) by a positive constant
\(\alpha\) and divide the slope \(\phi_1\) by the same parameter
\(\alpha\)? (b) What happens if \(\alpha\) is negative?

\vspace{1cm}

\textbf{Problem 3.5} Consider fitting the model in equation 3.1 using a
least squares loss function. Does this loss function have a unique minimum?
i.e., is there is a single “best” set of parameters?

\textit{Hint:} You don't need any math to answer. You can reason from the 
result in the previous problem.

\vspace{1cm}

\textbf{Problem 3.6} Write out the equations that define the network in
figure 3.11. There should be three equations to compute the three hidden
units from the inputs and two equations to compute the outputs from the
hidden units.

\vspace{1cm}

\textbf{Problem 3.7} Equations 3.11 and 3.12 define a general neural network
with \(D_i\) inputs, one hidden layer containing \(D\) hidden units, and
\(D_o\) outputs. Find an expression for the number of parameters in the
model in terms of \(D_i\), \(D\), and \(D_o\).

\vspace{1cm}

\end{document}  
