\documentclass[11pt, oneside]{article}   	% use "amsart" instead of "article" for AMSLaTeX format
\usepackage{geometry}                		% See geometry.pdf to learn the layout options. There are lots.
\geometry{letterpaper}                   	% ... or a4paper or a5paper or ... 
%\geometry{landscape}                		% Activate for rotated page geometry
\usepackage[parfill]{parskip}    	% Activate to begin paragraphs with an empty line rather than an indent
\usepackage{graphicx}				% Use pdf, png, jpg, or eps§ with pdflatex; use eps in DVI mode
								% TeX will automatically convert eps --> pdf in pdflatex		
\usepackage{amssymb}
\usepackage{amsmath}

%SetFonts

%SetFonts


\title{DL4DS Project Title}
\author{Team Member 1, Optional Team Member 2, Optional Team Member 3}
\date{April, 2024}		% Activate to display a given date or no date

\begin{document}
\maketitle
\begin{abstract}
A short abstract of what your project is about. This is your short "elevator pitch"
about your project and why it is interesting. This should be no
more than 3 sentences. Typically you write this last.
\end{abstract}

\section*{Introduction}

\textit{This is a template for your project final report. It's also pretty
standard template used for conference and journal publications. 
\textbf{Remove all these italicized instructions before submitting.}}

\textit{The introduction is where you motivate the problem you are trying
to solve. You should give a brief overview of the problem and why it is
important. You should also give a brief overview of your approach and
why it is interesting.}

\textit{You should also include a link to your github repository and have enough
documentation there to make your work reproducible.}

Introduction text here.

\section*{Related Work}

\textit{This is where you give a brief overview of any prior work by
others (or yourself) that is relevant to the problem and solution you
are proposing. Cite any papers using the citation and bibliography syntax illustrated
below.}

\textit{A comment on related work. You may find a paper or project that
directly solves the problem you are proposing. Did they also release code and
models? If not, is there value in reproducing their results and releasing
code and the model? 
If they did release the code and the model, is it possible to build
on their work directly and improve it?}

Related work text here. This also shows how put a single 
citation \cite{citation_key1} or also multiple citations
\cite{citation_key1,citation_key2}.  The bibliography is embedded
in this \LaTeX \hspace*{1pt} file.


\section*{Approach (or Methodology)}

\textit{This is where you describe your  solution to the problem.
Elaborate on the benefits of your approach.}

Proposed work text here. 

Here is also an example of how to render math inline
such as \(f(x) = \phi_0 + \phi_1 \cdot x\). Or you can render it as a block

\[
f(x) = \phi_0 + \phi_1 \cdot x
\]

or numbered
\begin{equation}
f(x) = \phi_0 + \phi_1 \cdot x.
\end{equation}

\section*{Datasets}

\textit{Describe the datasets that you used to train and evaluate your
models. Or if you are doing a dataset project, describe the dataset you
created. If it is a theoretical
or algorithmic project, describe any datasets that your theory or algorithm
may be applicable to.}

Dataset text here.

\section*{Evaluation Results}

\textit{Describe your evaluation results. What metrics
will did you use? What baseline from an existing solutions can you compare to?}

Evalutation text here.


\section*{Conclusion}

\textit{Summarize your project, results and contributions. Describe any
future work that you or someone else would do if they continued the project.}

Conclusion text here.

\begin{thebibliography}{9}

\bibitem{citation_key1}
Author1. 
\textit{Title1}.
Publisher, Year.

\bibitem{citation_key2}
Author2. 
\textit{Title2}.
Publisher, Year.

\end{thebibliography}

\end{document} 