\documentclass[11pt, oneside]{article}   	% use "amsart" instead of "article" for AMSLaTeX format
\usepackage{geometry}                		% See geometry.pdf to learn the layout options. There are lots.
\geometry{letterpaper}                   	% ... or a4paper or a5paper or ... 
%\geometry{landscape}                		% Activate for rotated page geometry
\usepackage[parfill]{parskip}    	% Activate to begin paragraphs with an empty line rather than an indent
\usepackage{graphicx}				% Use pdf, png, jpg, or eps§ with pdflatex; use eps in DVI mode
								% TeX will automatically convert eps --> pdf in pdflatex		
\usepackage{amssymb}
\usepackage{amsmath}

%SetFonts

%SetFonts


\title{DL4DS Project Title}
\author{Team Member 1, Optional Team Member 2, Optional Team Member 3}
\date{January 26, 2024}		% Activate to display a given date or no date

\begin{document}
\maketitle
\begin{abstract}
A short abstract of what your proposal is about. This is your elevator pitch
for what this project is about and why it is interesting. This should be no
more than 3 sentences. Typically you write this last.
\end{abstract}

\section*{Introduction}

\textit{This is a template for your project proposal which, by the way, 
is very similar to the template for your final report. It's also pretty
standard template used for conference and journal publications. For the
project proposal, however, your context in each section will be shorter
and preliminary. In other words it is likely to change for the final
report.  
\textbf{Remove all these italicized instructions before submitting.}}

\textit{The introduction is where you motivate the problem you are trying
to solve. You should give a brief overview of the problem and why it is
important. You should also give a brief overview of your approach and
why it is interesting.}

Introduction text here.

\section*{Related Work}

\textit{This is where you give a brief overview of any prior work by
others (or yourself) that is relevant to the problem and solution you
are proposing. For the proposal you should do a preliminary literature search
and cite any papers using the citation and bibliography syntax illustrated
below.}

\textit{A comment on related work. You may find a paper or project that
directly solves the problem you are proposing. Did they also release code and
models? If not, is there value in reproducing their results and releasing
code and the model? 
If they did release the code and the model, is it possible to build
on their work directly and improve it?}

Related work text here. This also shows how put a single 
citation \cite{citation_key1} or also multiple citations
\cite{citation_key1,citation_key2}.  The bibliography is embedded
in this \LaTeX \hspace*{1pt} file.


\section*{Proposed Work}

\textit{This is where you describe your proposed solution to the problem.
This is just the project proposal so at this point this is your best
guess, but you should try to elaborate on why you think your approach is
a promising one. Of course this will all evolve as you progress.}

Proposed work text here. 

Here is also an example of how to render math inline
such as \(f(x) = \phi_0 + \phi_1 \cdot x\). Or you can render it as a block

\[
f(x) = \phi_0 + \phi_1 \cdot x
\]

or even numbered
\begin{equation}
f(x) = \phi_0 + \phi_1 \cdot x.
\end{equation}

\section*{Datasets}

\textit{As part of the feasibility assessment for your project proposal
you should identify datasets that can use to train and evaluate your
models. Or if you are doing a dataset project, describe the dataset you
want to create and the approximate steps and timeline. If it is a theoretical
or algorithmic project, dscribe any datasets you may need in support of
that.}

Dataset text here.

\section*{Evaluation}

\textit{Describe how you will evaluate your proposed solution. What metrics
will you use? What kind of results do you expect to see? Will you be able
to establish a baseline from existing solutions?}

Evalutation text here.

\section*{Timeline}

\textit{This is a time bound project given the entire course is 14 weeks. 
The proposal deadline is about 3-4 weeks into the course, so you have
about 10-11 weeks to complete your project if you start after the proposal
deadline. You should break this down into
milestones and deliverables. Note that you can set discovery based milestones if
subsequent steps depend on the results of previous steps. The important
goal is to gain confidence in the feasibility of the project.}

Timeline text here.

\section*{Conclusion}

\textit{Summarize your proposal and what you hope to accomplish.}

Conclusion text here.

\begin{thebibliography}{9}

\bibitem{citation_key1}
Author1. 
\textit{Title1}.
Publisher, Year.

\bibitem{citation_key2}
Author2. 
\textit{Title2}.
Publisher, Year.

\end{thebibliography}

\end{document} 